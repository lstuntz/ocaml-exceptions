\documentclass{article}

\usepackage{microtype}

\usepackage{amsmath}
\usepackage{mathtools}
\usepackage{hyperref}

\title{CS225 Spring 2018---Final Project Proposal}
\author{
  Lindsey Stuntz \\ \small{\texttt{git:@lstuntz}}
}
\date{\today}

\begin{document}
\maketitle

\section*{Project: Exceptions}

I propose to implement a mechanism to support functions that throw exceptions 
and those that catch/handle exceptions in OCamL.  A semantics and type 
checker for this extended language will also be implemented.

\paragraph{Base Language}

I will work with simply typed lambda calculus with booleans, natural numbers,
let-binding, and products as the base language.

\paragraph{Extended Language}

I will extend this language with the features to catch, raise and handle exceptions.
This consists of three new terms:
\begin{enumerate}
\item A runtime-error, written \texttt{error}
\item An error handler, written \texttt{try}  $t$ \texttt{with} $t$
\item An error raiser, written \texttt{raise}  $t$
\end{enumerate}
and the implementation of small-step and typing rules to correspond with all of the 
above terms. Specific rules that will be modeled can be found in Pierce's 
\emph{Types and Programming Languages} Chapter 14.

\paragraph{Applications}

Exceptions have applications in real-world programming whenever a function needs to 
signal to its caller that it is unable to perform the requested task. This can be the 
result of any number of issues such as a calculation involving division by zero or
an arithmetic overflow, an out of bounds array index, a lookup key missing from a
dictionary, a file that cannot be found or opened, some event such as the system
running out of memory or the user killing the process, etc.

OCamL does handle errors through exceptions in the standard library. Exceptions
belong to an extensible sum type, \texttt{exn}, and are handled with the constuct
\begin{verbatim}
try . . .  with . . .
\end{verbatim}
The biggest issue with exceptions as OCamL handles them is that they do not appear 
in types. An example of where this becomes problematic comes when using functions 
like \texttt{Map.S.find} or \texttt{List.hd} are not total functions and may fail, but this can
only be found by reading the documentation.

\paragraph{Project Goals}

For this project, I plan to complete:
\begin{enumerate}
\item The following three mechanisms
	\begin{enumerate}
	\item A mechanism that causes the program to abort when it encounters an exception, 
	\item one for trapping and recovering from exceptions, 
	\item and finally one to refine the previous two mechanisms and allow for extra 
	programmer-specified data to be passed between exception sites and handlers.
	\end{enumerate}
\item A small-step semantics for exception handling
\item A type checker for exception handling
\item A formalized comparison of the exception handling that OCamL 
  already supports in the standard library against my extended computational
  model
\end{enumerate}

\paragraph{Expected Challenges}

The ambiguity, or flexibility, involved in the typing of the term \texttt{error} breaks the property 
that every typable term has a unique type. This can be dealt with in various ways and I expect 
some challenges to arise in considering the different possible implementations to deal with the
typechecking of \texttt{error} and determining the best choice for the given project.

A possible extension of this project that I would like to include provided that I implement 
all other aspects of exception handling outlined in a timely manner would be to encode 
exceptions using monads. I expect this to be a slight challenge as I have never learned 
about or worked with monads before and so would need to do outside research in order 
to add this implementation.

\paragraph{Timeline and Milestones}

By the checkpoint I hope to have completed:
\begin{enumerate}
\item A prototype implementation of the first two mechanism regarding exception
handling (program abortion and trapping and recovering)
\item An implementation for the small-step semantics and type-checker for the
two aforementioned exception mechanisms
\item A suite of test-cases for the two implemented exception mechanisms
\item One medium-sized program encoded in the language which demonstrates a
  real-world application of the language and how different types of exceptions could 
  come up and are handled
\end{enumerate}

\noindent
By the final project draft I hope to have completed:
\begin{enumerate}
\item The full implementation of exception handling, refining the two mechanisms 
previously implemented and adding in support to allow programmer-specific data passing
\item The full implementation of small-step semantics and type checking
\item A fully comprehensive test suite, with all tests passing
\item The medium-sized program running through all implementations of exception 
handling that have been implemented
\item A draft writeup that explains the on-paper formalism of our
  implementation as well a comparison of the exception handling that OCamL 
  already supports in the standard library against my extended computational
  model
\item A draft of a presentation with 5 slides as the starting point for our
  in-class presentation
\end{enumerate}

\noindent
By the final project submission I hope to have completed:
\begin{enumerate}
\item The final writeup and presentation
\item Any remaining implementation work that was missing in the final project
  draft
\end{enumerate}

\paragraph{Github Repository}

\href{https://github.com/lstuntz/ocaml-exceptions}{ocaml-extensions}

\paragraph{Resources Used for this Proposal}
\paragraph{}
Pierce, Benjamin C. \emph{Types and Programming Languages}. MIT Press, 2002.
\paragraph{}
``Error Handling." \emph{OCaml}, \href{https://ocaml.org/learn/tutorials/error_handling.html}{ocaml.org/learn/tutorials/error\_handling.html}.

\end{document}
